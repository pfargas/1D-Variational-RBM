\documentclass{book}
%--------------------------------------Paquets--------------------------------------
\usepackage{amsmath}
\usepackage{graphicx}
\usepackage{cancel}
\usepackage{multirow}
\usepackage{array}
\usepackage{verbatim} %Per fer comentaris llargs
\usepackage[margin=2cm]{geometry}
\usepackage{ wasysym }
\usepackage{ gensymb }
\usepackage{ragged2e}
\usepackage{multicol}
\usepackage{amssymb}
\usepackage{physics}




%--------------------------------------Comandes noves--------------------------------------


\newcommand{\dparcial}[2]{\frac{\partial #1}{\partial #2}}
\newcommand{\E}[1]{\cdot 10^{#1}}

\renewcommand{\baselinestretch}{1.125}


\title{Variational RBM}
\author{Pau Fargas Reixats}
\date{January 2025}

\begin{document}

\maketitle

\chapter{RBM ansatz for the 1D QHO}

Hamiltonian:

\[H=-\frac{1}{2}\partial_x^2+\frac{1}{2}x^2\]

Ansatz:

\[\Psi(x)=\sum_{\{\vec{h}\}}e^{xb+\vec{c}'\vec{h}+x\vec{W}\vec{h}}\]

Where $\{\vec{h}\}$ means the sum over all the possible combinations of $\vec{h}$. Taking $h_i=\{0,1\}\ i = 1,\dots, M$ the wavefunction can be written as:

\[\Psi(x)=e^{xb}\prod_{j=1}^M(1+e^{c_j+xW_j})\]

And we define the function $\Gamma(x)=\prod_{j=1}^M(1+e^{c_j+xW_j})$ so that the wavefunction is written as:

\[\Psi(x)=e^{xb}\Gamma(x)\]

\section{Energy}

Variational energy:

\[E=\frac{\bra{\Psi}H\ket{\Psi}}{\bra{\Psi}\ket{\Psi}}=\frac{\bra{\Psi}T\ket{\Psi}}{\bra{\Psi}\ket{\Psi}}+\frac{\bra{\Psi}V\ket{\Psi}}{\bra{\Psi}\ket{\Psi}}\]

Then:

\[\braket{\Psi}{\Psi}=\int dx |\Psi(x)|^2=\int dx e^{2xb}\Gamma^2(x)\]

\[\langle{T}\rangle_\Psi=-\frac12\int dx \Psi(x)\partial^2_x\Psi(x)\]

\[\langle{V}\rangle_\Psi=\frac12\int dx \Psi^2(x)x^2\]

Let's compute $\partial^2_x\Psi(x)$:

\[\partial_x ^2\Psi = \sum_{\{\vec{h}\}}e^{x(b+\vec{W}\vec{h})+\vec{c}'\vec{h}}(b+\vec{W}\vec{h})^2\]

And expanding the squared factor, we have three terms:

\begin{itemize}
    \item $\eta_1=b^2\sum_{\{\vec{h}\}}e^{x(b+\vec{W}\vec{h})+\vec{c}'\vec{h}}$
    \item $\eta_2=2b\sum_{\{\vec{h}\}}e^{x(b+\vec{W}\vec{h})+\vec{c}'\vec{h}}\vec{W}\vec{h}$
    \item $\eta_3=\sum_{\{\vec{h}\}}e^{x(b+\vec{W}\vec{h})+\vec{c}'\vec{h}}(\vec{W}\vec{h})^2$
\end{itemize}

TO WRITE

Defining $\sigma_i=\sigma(c_i+W_i x) = \frac{1}{1+e^{-(c_i+W_i x)}}$, and $\vec{\sigma}=(\sigma_1,\sigma_2,\dots,\sigma_M)$, the energy reads:

\[E=\frac{\int dx \Psi^2(x)\biggl\{-\frac{1}{2}\biggl[b^2+2b(\vec{W}\vec{\sigma})+\sum_{ij}W_iW_j(\delta_{ij}\sigma_j+\sigma_i\sigma_j(1-\delta_{ij}))\biggr]+\frac{1}{2}x^2\biggr\}}{\int dx \Psi^2}\]

\section{Gradient}

Now we have to compute $\grad_\theta E=(\partial_b+\sum_k[\partial_{c_k}+\partial_{W_k}])E$:

\[\grad_\theta E=\grad_\theta \bigg(\frac{\bra{\Psi}T\ket{\Psi}}{\bra{\Psi}\ket{\Psi}}+\frac{\bra{\Psi}V\ket{\Psi}}{\bra{\Psi}\ket{\Psi}}\bigg)=\frac{\grad_\theta(\ev{T}{\Psi}+\ev{V}{\Psi})}{\braket{\Psi}{\Psi}}-\frac{\ev{T}{\Psi}+\ev{V}{\Psi}}{\braket{\Psi}{\Psi}^2}\grad_\theta\braket{\Psi}{\Psi}\]

Then, the problem is to compute $\grad_\theta(\ev{T}{\Psi}+\ev{V}{\Psi})$ and $\grad_\theta\braket{\Psi}{\Psi}$. Starting from the last one:

\[\grad_\theta\braket{\Psi}{\Psi}=\grad_\theta\int dx\Psi^2=\int dx\grad_\theta\Psi^2=2\int dx \Psi(x)\grad_\theta \Psi\]

Finally:

\[\grad_\theta(\ev{T}{\Psi}+\ev{V}{\Psi})=\frac{1}{2}\grad_\theta \int dx \Psi^2(x)\biggl\{-\biggl[b^2+2b(\vec{W}\vec{\sigma})+\sum_{ij}W_iW_j(\delta_{ij}\sigma_j+\sigma_i\sigma_j(1-\delta_{ij}))\biggr]+x^2\biggr\}\]

\begin{align*}
    \frac{1}{2}\grad_\theta \int dx \Psi^2(x)\underbrace{\biggl\{-\biggl[b^2+2b(\vec{W}\vec{\sigma})+\sum_{ij}W_iW_j(\delta_{ij}\sigma_j+\sigma_i\sigma_j(1-\delta_{ij}))\biggr]+x^2\biggr\}}_\xi=\frac{1}{2}\int dx \grad_\theta (\Psi^2 \xi)=\\
    \frac{1}{2}\int dx\grad_\theta (\Psi ^2)\xi+\Psi^2\grad_\theta \xi=\frac{1}{2}\int dx 2\Psi\grad_\theta(\Psi)\xi+\grad_\theta \xi
\end{align*}

Useful derivatives:

\[\partial_b\Gamma(x)=0\]

\[\partial_{c_k}\Gamma(x)=\Gamma(x)\sigma_k\]

\[\partial_{W_k}\Gamma(x)=\Gamma(x)\sigma_k x\]

\[\frac{d\sigma(x)}{dx}=\sigma(x)(1-\sigma(x))\]

\[\partial_{c_k}\sigma_i=\sigma_i(1-\sigma_i)\partial_{c_k}(c_i+W_ix)=\sigma_i(1-\sigma_i)\delta_{ik}\]

\[\partial_{W_k}\sigma_i=\sigma_i(1-\sigma_i)\partial_{W_k}(c_i+W_ix)=\sigma_i(1-\sigma_i)x\delta_{ik}\]

\subsection{$\grad_\theta\xi$}

\paragraph{w.r.t. $b$}

\[\partial_b\xi=\partial_b\biggl\{-\biggl[b^2+2b(\vec{W}\vec{\sigma})+\sum_{ij}W_iW_j(\delta_{ij}\sigma_j+\sigma_i\sigma_j(1-\delta_{ij}))\biggr]+x^2\biggr\}=-2(b+\vec{W}\vec{\sigma})\]

\paragraph{w.r.t. $c_k$}

\[\partial_{c_k}\xi=\partial_{c_k}\biggl\{-\biggl[\cancel{b^2}+2b(\vec{W}\vec{\sigma})+\sum_{ij}W_iW_j(\delta_{ij}\sigma_j+\sigma_i\sigma_j(1-\delta_{ij}))\biggr]+\cancel{x^2}\biggr\}=-\partial_{c_k}\bigg[2b\vec{W}\vec{\sigma}+\sum_{ij}W_iW_j(\delta_{ij}\sigma_j+\sigma_i\sigma_j(1-\delta_{ij}))\bigg]\]

\[-2b\sum_i W_i\partial_{c_k}\sigma_i-\sum_{ij}W_iW_j\partial_{c_k}(\delta_{ij}\sigma_j+\sigma_i\sigma_j(1-\delta_{ij}))=-2bW_k\sigma_k(1-\sigma_k)-W_k ^2\sigma_k(1-\sigma_k)-\sum_{i\neq j}W_iW_j\underbrace{\partial_{c_k}(\sigma_i\sigma_j)}_{(\delta_{ik}\sigma_j+\delta_{jk}\sigma_i)\sigma_k(1-\sigma_k)}\]

\[-2bW_k\sigma_k(1-\sigma_k)-W_k ^2\sigma_k(1-\sigma_k)-\sum_{i\neq j}W_iW_j{(\delta_{ik}\sigma_j+\delta_{jk}\sigma_i)\sigma_k(1-\sigma_k)}\]

\[\partial_{c_k}\xi=-2bW_k\sigma_k(1-\sigma_k)-W_k ^2\sigma_k(1-\sigma_k)-2\sum_{i\neq k}W_iW_k{\sigma_i\sigma_k(1-\sigma_k)}\]

\paragraph{w.r.t. $W_k$}

\[\partial_{W_k}\xi=\partial_{c_k}\biggl\{-\biggl[\cancel{b^2}+2b(\vec{W}\vec{\sigma})+\sum_{ij}W_iW_j(\delta_{ij}\sigma_j+\sigma_i\sigma_j(1-\delta_{ij}))\biggr]+\cancel{x^2}\biggr\}=-\partial_{W_k}\bigg[2b\vec{W}\vec{\sigma}+\sum_{ij}W_iW_j(\delta_{ij}\sigma_j+\sigma_i\sigma_j(1-\delta_{ij}))\bigg]\]

\[-2b\sum_i\underbrace{\partial_{W_k}(W_i\sigma_i)}_{\partial_{W_k}(W_i)\sigma_i+W_i\partial_{W_k}\sigma_i}-\underbrace{\partial_{W_k}\sum_{ij}W_iW_j(\delta_{ij}\sigma_j+\sigma_i\sigma_j(1-\delta_{ij}))}_{\sum_{ij}\partial_{W_k}\bigg(W_iW_j(\delta_{ij}\sigma_j+\sigma_i\sigma_j(1-\delta_{ij}))\bigg)}\]

For the first term we have:

\[-2b\sum_i\bigg(\partial_{W_k}(W_i)\sigma_i+W_i\partial_{W_k}\sigma_i\bigg)=-2b\sum_i\bigg(\delta_{ik}\sigma_i+W_i\sigma_i(1-\sigma_i)x\delta_{ik}\bigg)=-2b(\sigma_k+W_k\sigma_k(1-\sigma_k)x)\]

And for the second (remembering the minus for after):

\begin{align*}
    \sum_{ij}\partial_{W_k}\bigg(W_iW_j(\delta_{ij}\sigma_j+\sigma_i\sigma_j(1-\delta_{ij}))\bigg) &= \sum_i\partial_{W_k}(W_i ^2\sigma_i)+\sum_{i\neq j}\partial_{W_k}(W_iW_j\sigma_i\sigma_j)\\&=\sum_i\bigg(\partial_{W_k}(W_i^2)\cdot \sigma_i+W_i^2\partial_{W_k}\sigma_i\bigg)+\sum_{i\neq j}\partial_{W_k}(W_iW_j\sigma_i\sigma_j)\\&=\sum_i\bigg(2W_i\delta_{ik}\sigma_i+W_i^2\delta_{ik}\sigma_k(1-\sigma_k)x\bigg)+\sum_{i\neq j}\partial_{W_k}(W_iW_j\sigma_i\sigma_j)\\&=2W_k\sigma_k+W_k^2\sigma_k(1-\sigma_k)x+\sum_{i\neq j}\partial_{W_k}(W_iW_j\sigma_i\sigma_j)
\end{align*}

Then, the non-diagonal reads:

\begin{align*}
    \partial_{W_k}(W_iW_j\sigma_i\sigma_j)&= \partial_{W_k}(W_i)W_j \sigma_i\sigma_j\to \delta_{ik}W_j\sigma_i\sigma_j \\&+\partial_{W_k}(W_j)W_i\sigma_i\sigma_j\to \delta_{jk}W_i\sigma_i\sigma_j\\ &+\partial_{W_k}(\sigma_i) W_iW_j\sigma_j \to \delta_{ik}x\sigma_k(1-\sigma_k)W_iW_j\sigma_j\\ &+\partial_{W_k}(\sigma_j)W_iW_j\sigma_i \to \delta_{jk}x\sigma_k(1-\sigma_k)W_iW_j\sigma_i
\end{align*}

And adding everything together:

\begin{align*}
    \sum_{i\neq j}\partial_{W_k}(W_iW_j\sigma_i\sigma_j)&=\sum_{i\neq j}\bigg(\delta_{ik}W_j\sigma_i\sigma_j+\delta_{jk}W_i\sigma_i\sigma_j+\delta_{ik}x\sigma_k(1-\sigma_k)W_iW_j\sigma_j+\delta_{jk}x\sigma_k(1-\sigma_k)W_iW_j\sigma_i\bigg)\\&=2\sum_{i\neq k}(W_i\sigma_i\sigma_k+W_i\sigma_i\sigma_k(1-\sigma_k)xW_k)\\&=2(\sigma_k+x\sigma_k(1-\sigma_k)W_k)\vec{W}\vec{\sigma}
\end{align*}

Finally, we regroup all terms together:

\[\sum_{ij}\partial_{W_k}\bigg(W_iW_j(\delta_{ij}\sigma_j+\sigma_i\sigma_j(1-\delta_{ij}))\bigg)=2W_k\sigma_k+W_k^2\sigma_k(1-\sigma_k)x+2(\sigma_k+x\sigma_k(1-\sigma_k)W_k)\vec{W}\vec{\sigma}\]

Finally:

\[\partial_{W_k}\xi=-\bigg[2b(\sigma_k+W_k\sigma_k(1-\sigma_k)x)+2W_k\sigma_k+W_k^2\sigma_k(1-\sigma_k)x+2(\sigma_k+x\sigma_k(1-\sigma_k)W_k)\vec{W}\vec{\sigma}\bigg]\]

\section{Gaussian RBM}

Let's try an ansatz of the shape:

\[\Psi(x)=\sum_{\{\vec{h}\}}e^{-\lambda^2x^2+xb+\vec{c}'\vec{h}+x\vec{W}\vec{h}}\to\Psi(x)=e^{-\lambda^2x^2+xb}\prod_{j=1}^M(1+e^{c_j+xW_j})\]

So that the energy of the RBM $E_{RBM}$ reads:

\[E_{RBM}=-\lambda^2x^2+xb+\bf{c}'\bf{h}+x\bf{W}\bf{h}\]

\[E=\frac{\int\psi^*(x)\hat{H}\psi(x) dx}{\int|\psi|^2}\]

Only numerator:

\[\int\psi^*(x)\hat{H}\psi(x) dx=\int |\psi|^2\underbrace{\frac{\hat{H}\psi(x)}{\psi(x)}}_{E_{loc}}dx=\int|\psi|^2\bigg(-\frac{1}{2\psi(x)}\frac{\partial^2}{\partial x^2}\psi(x)+\frac12 x^2\bigg)\]

Then, if $\bigg(\partial_x^2\psi(x)\bigg)_i\approx \frac{\psi_{i+1}-2\psi_i+\psi_{i-1}}{h^2}$:

\[\int|\psi|^2\frac{\partial_x^2\psi}{\psi}dx\approx \int |\psi|^2\frac{\psi_{i+1}-2\psi_i+\psi_{i-1}}{\psi_ih^2}dx\]

\section{Other way}

To compute $\langle T\rangle_\Psi$ we can integrate by parts:

\[\langle T\rangle_\Psi=-\frac{1}{2}\int^{+\infty}_{-\infty} dx \Psi^*(x)\partial_x^2\Psi(x) =-\frac{1}{2}\bigg\{ \eval{\Psi^*\partial_x \Psi}_{-\infty}^{+\infty}-\int_{-\infty}^{+\infty}dx(\partial_x\Psi^*)(\partial_x\Psi)\bigg\}\]

And now we can apply the fact that $\displaystyle\lim_{x\to\pm\infty}\Psi(x)=0$ so that we obtain:

\[\langle T\rangle_\Psi=\frac{1}{2}\int_{-\infty}^{+\infty}dx(\partial_x\Psi^*)(\partial_x\Psi) = \frac{1}{2}\int_{-\infty}^{+\infty}dx||\partial_x\Psi||^2\]


If $\Psi(x)\in\mathbb{R}$, then one can compute the kinetic expected value as:

\[\langle T\rangle_\Psi=\frac{1}{2}\int_{-\infty}^{+\infty}dx(\partial_x\Psi)^2\]

And the Hamiltonian expected value is then:

\[\langle\hat{H}\rangle=\frac{1}{2}\int_{-\infty}^{+\infty}dx\bigg(  \Psi ^*\big(\partial_x\Psi(x)\big)^2+x^2\Psi^2\bigg)\]

\paragraph{Importance sampling} Multiplying everything by $\frac{\Psi}{\Psi}$ we have:

\[\langle\hat{H}\rangle=\frac{1}{2}\int_{-\infty}^{+\infty}dx||\Psi||^2\underbrace{\bigg(\frac{||\partial_x\Psi||^2}{\Psi}+x^2\bigg)}_{E_{loc}(x)}\]

So that:

\[E_{loc}(x) = T_{loc}(x)+V_{loc}(x)\]

with (supposing $\hat{V}=V(x)$):

\[\begin{cases}
    T_{loc}(x) = \frac{||\partial_x\Psi(x)||^2}{\Psi(x)}\\
    V_{loc}(x) = V(x)
\end{cases}\]

\section{Metropolis Algorithm}

We want to sample from $\pi(x)$

Acceptance probability:

\[p = \min\bigg\{1,\frac{\pi(x')}{\pi(x)}\bigg\}\]

In statistical physics, $\pi(x) \propto e^{-\beta E(x)}$. In our case, $E = \langle\hat{H}\rangle$

\subsection{Parallel Tempering}

The acceptance probability of an exchange is defined as a Metropolis step such that %TODO

\end{document}
